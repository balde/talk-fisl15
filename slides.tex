\documentclass{beamer}
\usepackage[utf8]{inputenc} % codificacao de caracteres
\usepackage[T1]{fontenc}    % codificacao de fontes
\usepackage[brazil]{babel}  % idioma
\usepackage{listings}
\usetheme{Antibes}         % tema

\begin{document}
\title{balde: Desenvolvendo aplicações web em C}   
\author{Rafael G. Martins} 
\date{\today} 

\frame{\titlepage} 

%\frame{\frametitle{Tópicos}\tableofcontents} 


\section{Quem sou?!}
\frame{
    \frametitle{Quem sou?!}
    \begin{itemize}
        \item Engenheiro de controle e automação
        \item Desenvolvedor do Gentoo Linux
        \item Desenvolvedor Python no Titans Group
    \end{itemize}
}


\section{balde?!}
\frame{
    \frametitle{balde?!}
    \begin{itemize}
        \item O que quer dizer esse nome?!
            \pause
        \item Alguém ainda programa em C?!
            \pause
        \item Isso não deveria estar sendo feito em Python/PHP/Ruby/(insira sua linguagem hipster preferida aqui)?!
    \end{itemize}
}

\section{Como instalar?}
\frame{\frametitle{Como instalar?}
    \begin{itemize}
        \item \url{http://balde.io/}
            \pause
        \item \url{https://github.com/balde}
            \pause
        \item Poucas dependências: FastCGI toolkit, shared-mime-info, PEG.
            \pause
        \item Funciona com qualquer servidor WEB que suporte CGI/FastCGI.
            \pause
        \item
            \$ ./autogen.sh \&\& ./configure \&\& make \&\& sudo make install
    \end{itemize}
}

\section{Hello World}
\frame{\frametitle{Hello world}
    Live coding!
}

\section{Views}
\frame{\frametitle{Views}
    \begin{itemize}
        \item Views são chamadas pelo framework para responder a uma requisição.
            \pause
        \item Uma view recebe um contexto de aplicação e um contexto de requisição, e retorna um contexto de resposta.
            \pause
        \item A definição das rotas e métodos HTTP atendidos pela view é feita no momento do registro da view.
    \end{itemize}
}

\section{Contexto de aplicação}
\frame{\frametitle{Contexto de aplicação}
    \begin{itemize}
        \item Armazena toda a informação necessária durante todo o ciclo de vida da aplicação.
            \pause
        \item Registra todas as views e realiza o roteamento das URLs e métodos HTTP.
            \pause
        \item Pode armazenar dados arbitrários da aplicação, para serem facilmente acessados pelas views, se necessário.
    \end{itemize}
}

\section{Contexto de requisição}
\frame{\frametitle{Contexto de requisicao}
    \begin{itemize}
        \item Armazena toda a informação proveniente de uma requisição HTTP feita por um cliente.
            \pause
        \item Existem funções especiais para manipular os dados armazenados neste contexto.
    \end{itemize}
}

\section{Contexto de resposta}
\frame{\frametitle{Contexto de resposta}
    \begin{itemize}
        \item Deve ser construido na view para armazenar toda a informação necessária para construir a resposta da requisição HTTP.
            \pause
        \item Existem funções especiais para inserir os dados neste contexto.
    \end{itemize}
}

\section{Recursos estáticos}
\frame{\frametitle{Recursos estáticos}
    \begin{itemize}
        \item Recursos estáticos da aplicação, como arquivos CSS, imagens e scripts JavaScript, podem ser convertidos para codigo C e linkados com o binário final da aplicação.
            \pause
        \item Utiliza os objetos GResource da GLib e o utilitário glib-resources-compile.
            \pause
        \item Todos os recursos estáticos presentes na aplicação são registrados automaticamente num endpoint especifico (/static/).
    \end{itemize}
}

\section{Templates}
\frame{\frametitle{Templates}
    \begin{itemize}
        \item Engine de templates "logic-less".
            \pause
        \item É capaz de manipular variáveis, strings, inteiros, reais e chamadas simples de funções.
            \pause
        \item Todos os templates são convertidos em código C, que devem ser compilados e linkados com o binário final da aplicação.
    \end{itemize}
}

\section{Perguntas?!}
\frame{\frametitle{Perguntas?!}
    \begin{itemize}
        \item Email: \url{rafael@rafaelmartins.eng.br}
        \item Blog: \url{http://rafaelmartins.eng.br}
        \item IRC: rafaelmartins @ Freenode
    \end{itemize}
}

\end{document}
